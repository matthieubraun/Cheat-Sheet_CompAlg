\subsection*{Intégration numérique}
\subsubsection*{Méthode composite du trapèze}
\noindent
Pour les points $x_0, x_1, \dots, x_n$ avec $x_0=a$ et $x_n=b$.
\begin{equation}
    \left\{
    \begin{aligned}
         & T(h) = h\cdot \left[\frac{1}{2}\cdot f(a)+\sum_{j=1}^{n-1}f(x_j)+\frac{1}{2}\cdot f(b)\right] \\
         & h = \frac{b-a}{n}                                                                             \\
    \end{aligned}
    \right.
    \nonumber
\end{equation}
\subsubsection*{Méthode du point milieu}
\noindent
\begin{equation}
    \left\{
    \begin{aligned}
         & M(h) = h\cdot \sum_{j=0}^{n-1}f(x_{j+\sfrac{1}{2}}) \\
         & x_{j+\sfrac{1}{2}} = a + (j + \sfrac{1}{2})\cdot h
    \end{aligned}
    \right.
    \nonumber
\end{equation}
Il existe la relation suivant entre les deux méthodes :
\begin{equation}
    T\left(\frac{h}{2}\right) = \frac{1}{2}\cdot [T(h) + M(h)]
    \nonumber
\end{equation}
\subsubsection*{Méthode de Simpson simple}
\noindent
\begin{equation}
    S = \frac{h}{3}\cdot \left[f(a)+4\cdot f\left(\frac{a+b}{2}\right)+f(b)\right]
    \nonumber
\end{equation}
\subsubsection*{Méthode de Simpson composée}
\noindent
\begin{equation}
    \left\{
    \begin{aligned}
        S_c & = \frac{h}{3}\cdot \left(f(a)+4\cdot f(x_1)+f(b)\right.     \\
            & \quad+2\cdot \sum_{k=1}^{n-1}[f(x_{2k})+2\cdot f(x_{2k+1})] \\
        x_n & = a + n\cdot h
    \end{aligned}
    \right.
    \nonumber
\end{equation}
\subsubsection*{Méthode Newton-Cotes}
\noindent
\begin{equation}
    \left\{
    \begin{aligned}
         & Q_n=\int_a^b P_n(x)\cdot dx = \sum_{j=0}^n w_j\cdot f(x_j) \\
         & w_j = \int_a^b \prod_{\substack{k=0                        \\ k\neq j}}^n \frac{x-x_k}{x_j-x_k}\cdot dx
    \end{aligned}
    \right.
    \nonumber
\end{equation}
Exemple avec $n=1$, $x_0=a$ et $x_1=b$ parce que ça veut juste rien dire :
\begin{equation}
    Q_1 = \frac{h}{2}\cdot[f(x_0)+f(x_1)]
    \nonumber
\end{equation}
$n=2$, $x_0=a$ et $x_2=b$
\begin{equation}
    Q_2 = \frac{h}{3}\cdot[f(x_0)+4\cdot f(x_1)+f(x_2)]
    \nonumber
\end{equation}
$n=3$, $x_0=a$ et $x_3=b$
\begin{equation}
    Q_3 = \frac{3\cdot h}{8}\cdot[f(x_0)+3\cdot f(x_1)+3\cdot f(x_2)+f(x_3)]
    \nonumber
\end{equation}
$n=4$, $x_0=a$ et $x_4=b$
\begin{equation}
    Q_4 = \frac{2\cdot h}{45}\cdot[7\cdot f(x_0)+32\cdot f(x_1)+12\cdot f(x_2)+32\cdot f(x_3)+7\cdot f(x_4)]
    \nonumber
\end{equation}
\subsubsection*{Méthode de Romberg}
\noindent
Pour les $T_{i,0}$, on utilise la méthode du trapèze. Pour le reste :
\begin{equation}
    \left\{
    \begin{aligned}
         & T_{i,0} = T\left(\frac{h}{2^n}\right)                  \\
         & T_{i,j} = \frac{4^j\cdot T_{i,j-1}-T_{i-1,j-1}}{4^j-1} \\
    \end{aligned}
    \right.
    \nonumber
\end{equation}
Ce qui donne le schéma suivant :
\begin{equation}
    \begin{tabular}{ccc}
        $T_{0,0}$ &           &           \\
        $T_{1,0}$ & $T_{1,1}$ &           \\
        $T_{2,0}$ & $T_{2,1}$ & $T_{2,2}$ \\
    \end{tabular}
    \nonumber
\end{equation}
Afin d'avoir un comportement de convergence, il faut que $f(x)$ soit
$2n + 2$ fois continûment dérivable
\subsubsection*{Méthode de Gauss}
\noindent
Juste genre vraiment j'ai rien compris là.
