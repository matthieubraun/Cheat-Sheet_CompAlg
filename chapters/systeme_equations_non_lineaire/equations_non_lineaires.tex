\subsection*{Équations non linéaires}
\noindent
\subsubsection*{Méthode de la bissection}
\noindent
On sait que le $f(x)$ coupe l'axe des $x$ entre $a$ et $b$. $a_0 = a$, $b_0 = b$
\begin{equation}
    \begin{cases}
        a_{k+1} = a_k, b_{k+1} = x_k & \text{si } f(x_k)\cdot f(a_k) < 0 \\
        a_{k+1} = x_k, b_{k+1} = b_k & \text{si } f(x_k)\cdot f(b_k) < 0 \\
        x_{k+1} = \frac{a_{k+1}+b_{k+1}}{2}
    \end{cases}
    \nonumber
\end{equation}
\subsubsection*{Méthode de la sécante}
\noindent
\begin{equation}
    x_{k+1} = x_k - \frac{f(x_k)\cdot (x_k-x_{k-1})}{f(x_k)-f(x_{k-1})}
    \nonumber
\end{equation}
\subsubsection*{Méthode de Newton}
\noindent
\begin{equation}
    x_{k+1} = x_k - \frac{f(x_k)}{f^{\prime}(x_k)}
    \nonumber
\end{equation}
\subsubsection*{Point fixe}
\noindent
Voir les autre section où on met les équations sous forme de point fixe.
\begin{equation}
    x_{k+1} = F(x_k)
    \nonumber
\end{equation}
\subsubsection*{Erreurs}
\noindent
Erreur a priori :
\begin{equation}
    E_k = \frac{L^k}{1-L}\cdot |x_1-x_0|
    \nonumber
\end{equation}
Erreur a posteriori :
\begin{equation}
    E_k = \frac{L}{1-L}\cdot |x_k-x_{k-1}|
    \nonumber
\end{equation}
Avec :
\begin{equation}
    L = \max_{x\in [a,b]}|F^{\prime}(x)| < 1
    \nonumber
\end{equation}
\subsubsection*{Vitesse de convergence}
\noindent
Plus $P$ est grand, plus la vitesse de convergence est rapide.
$r$ est la racine de $f(x)$.
\noindent
\begin{equation}
    |x_{k+1}-r| \leq C\cdot |x_k-r|^p
    \nonumber
\end{equation}

