\subsection*{Algèbre linéaire numérique}
\noindent
On peut réécrire un système d'équations linéaires sous la forme $A\cdot x=b$.
\begin{equation}
    \begin{pmatrix}
        a_{11} & a_{12} & \cdots & a_{1n} \\
        a_{21} & a_{22} & \cdots & a_{2n} \\
        \vdots & \vdots & \ddots & \vdots \\
        a_{m1} & a_{m2} & \cdots & a_{mn}
    \end{pmatrix}
    \cdot
    \begin{pmatrix}
        x_1    \\
        x_2    \\
        \vdots \\
        x_n
    \end{pmatrix}
    =
    \begin{pmatrix}
        b_1    \\
        b_2    \\
        \vdots \\
        b_m
    \end{pmatrix}
    \nonumber
\end{equation}
Pour les exemples on prend toujours le système suivant :
\begin{equation}
    \begin{pmatrix}
        1 & 2 & 3 \\
        4 & 5 & 6 \\
        7 & 8 & 9
    \end{pmatrix}
    \cdot
    \begin{pmatrix}
        x_1 \\
        x_2 \\
        x_3
    \end{pmatrix}
    =
    \begin{pmatrix}
        1 \\
        2 \\
        3
    \end{pmatrix}
    \nonumber
\end{equation}
\subsubsection*{Méthode de Gauss}
\noindent
\begin{equation}
    \left(\begin{array}{ccc|c}
        1 & 2 & 3 & 1 \\
        4 & 5 & 6 & 2 \\
        7 & 8 & 9 & 3
    \end{array}\right)
    \rightarrow
    \left(\begin{array}{ccc|c}
        1 & 2  & 3   & 1  \\
        4 & 4  & 6   & 2  \\
        0 & -6 & -12 & -4
    \end{array}\right)
    \nonumber
\end{equation}
\begin{equation}
    \rightarrow
    \left(\begin{array}{ccc|c}
        1 & 2 & 3  & 1  \\
        4 & 5 & 6  & 2  \\
        0 & 0 & -3 & -1
    \end{array}\right)
    \rightarrow
    \left(\begin{array}{ccc|c}
        1 & 2  & 3  & 1  \\
        0 & -3 & -6 & -2 \\
        0 & 0  & -3 & -1
    \end{array}\right)
    \nonumber
\end{equation}
Puis on remonte et on trouve $x_3=\sfrac{-4}{-6}=\sfrac{2}{3}$, $x_2=\dots$
\subsubsection*{Décomposition LU}
\noindent
Le but de la décomposition LU est de factoriser la matrice $A$ en
deux matrices $L$ et $U$ telles que $A=L\cdot U$. Elle ne s'applique qu'aux
matrices carrées.\\
\begin{equation}
    \begin{pmatrix}
        a_{11} & a_{12} & a_{13} \\
        a_{21} & a_{22} & a_{23} \\
        a_{31} & a_{32} & a_{33}
    \end{pmatrix}
    =
    \begin{pmatrix}
        1      & 0      & 0 \\
        l_{21} & 1      & 0 \\
        l_{31} & l_{32} & 1
    \end{pmatrix}
    \cdot
    \begin{pmatrix}
        u_{11} & u_{12} & u_{13} \\
        0      & u_{22} & u_{23} \\
        0      & 0      & u_{33}
    \end{pmatrix}
    \nonumber
\end{equation}
Les coefficients sont calculés comme suit :
\begin{equation}
    \begin{aligned}
         & a_{11}= 1 \cdot u_{11} + 0 \cdot 0 + 0 \cdot 0           \\
         & a_{12}= 1 \cdot u_{12} + 0 \cdot u_{22} + 0 \cdot 0      \\
         & a_{13}= 1 \cdot u_{13} + 0 \cdot u_{23} + 0 \cdot u_{33} \\
         & a_{21}= l_{21} \cdot u_{11} + 1 \cdot 0 + 0 \cdot 0      \\
         & a_{22}= l_{21} \cdot u_{12} + 1 \cdot u_{22} + 0 \cdot 0 \\
         & \dots
    \end{aligned}
    \nonumber
\end{equation}
Puis comme on connait les $a_{ij}$ on peut trouver les $l_{ij}$ et les $u_{ij}$.
Ensuite on résout le système $L\cdot \overrightarrow{y}=\overrightarrow{b}$ et
$U\cdot \overrightarrow{x}=\overrightarrow{y}$ :
\begin{equation}
    \begin{pmatrix}
        1      & 0      & 0 \\
        l_{21} & 1      & 0 \\
        l_{31} & l_{32} & 1
    \end{pmatrix}
    \cdot
    \begin{pmatrix}
        y_1 \\
        y_2 \\
        y_3
    \end{pmatrix}
    =
    \begin{pmatrix}
        1 \\
        2 \\
        3
    \end{pmatrix}
    \nonumber
\end{equation}
\begin{equation}
    \begin{pmatrix}
        u_{11} & u_{12} & u_{13} \\
        0      & u_{22} & u_{23} \\
        0      & 0      & u_{33}
    \end{pmatrix}
    \cdot
    \begin{pmatrix}
        x_1 \\
        x_2 \\
        x_3
    \end{pmatrix}
    =
    \begin{pmatrix}
        y_1 \\
        y_2 \\
        y_3
    \end{pmatrix}
    \nonumber
\end{equation}
Il faut faire attention si la première valeur de la matrice est très petite,
il faut permuter les lignes. Sinon, on peut avoir des erreurs d'arrondi à cause
des chiffres significatifs. 3 chiffres dans la mantisse signifie
: $1.234\cdot 10^6=1.230\cdot 10^6$\\
Attention si on permute des lignes, il faut aussi permuter les valeurs de $b$ :
\begin{equation}
    PA\overrightarrow{x}=P\overrightarrow{b}
    \nonumber
\end{equation}
\begin{equation}
    L\underbrace{U\overrightarrow{x}}_{\overrightarrow{y}}=P\overrightarrow{b}
    \nonumber
\end{equation}
\subsubsection*{Transformation de Householder}
\noindent
La transformation de Householder permet de transformer une matrice $A$ en une
matrice $R$ triangulaire supérieure. Pour cela il faut recalculer les vecteurs
$\overrightarrow{x}$ et les matrices $H$ pour chaque colonne de la matrice $A$.
\begin{equation}
    H\cdot \overrightarrow{x}=-\rho\cdot \overrightarrow{e}_1
    \nonumber
\end{equation}
Pour une matrice 3x3 :
\begin{equation}
    \begin{tabular}{cc}
        $\overrightarrow{x_1}=
            \begin{pmatrix}
                x_1 \\
                x_2 \\
                x_3
            \end{pmatrix}
            \rightarrow
        \begin{pmatrix}
                -\rho_1 \\
                0       \\
                0
            \end{pmatrix}$ &
        $\overrightarrow{x_2}=
            \begin{pmatrix}
                x_1 \\
                x_2 \\
                x_3
            \end{pmatrix}
            \rightarrow
            \begin{pmatrix}
                \alpha  \\
                -\rho_2 \\
                0
            \end{pmatrix}$
    \end{tabular}
    \nonumber
\end{equation}
\begin{equation}
    \overrightarrow{x_3}=
    \begin{pmatrix}
        x_1 \\
        x_2 \\
        x_3
    \end{pmatrix}
    \rightarrow
    \begin{pmatrix}
        \beta  \\
        \gamma \\
        \delta
    \end{pmatrix}
    \nonumber
\end{equation}
\begin{equation}
    \rho = \text{sign}(x_1)\cdot \left \|  \overrightarrow{x}\right \|_2
    \nonumber
\end{equation}
\begin{equation}
    \overrightarrow{v} = \overrightarrow{x}+\rho\cdot \overrightarrow{e}_1
    \nonumber
\end{equation}
\begin{equation}
    \gamma = \frac{\left \| \overrightarrow{v} \right \|_2^2}{2}=\rho\cdot v_1
    \nonumber
\end{equation}
Et donc la matrice $H$ se calcul :
\begin{equation}
    H=I-\frac{\overrightarrow{v}\cdot \overrightarrow{v}^T}{\gamma}
    \nonumber
\end{equation}
\subsubsection*{Décomposition QR}
\noindent
Le but de la décomposition QR est de factoriser la matrice $A$ en
deux matrices $Q$ et $R$ telles que $A=Q\cdot R$. Elle s'applique à toutes les
matrices.\\
Exemple pour une matrice 3x3 :\\
\textbf{1ère colonne :}
calculer $H_1$ avec $\overrightarrow{x_1}$ puis :
\begin{equation}
    A_1=H_1\cdot A = \begin{pmatrix}
        -\rho_1 & \alpha & \beta \\
        0       & \star  & \star \\
        0       & \star  & \star
    \end{pmatrix}
    \nonumber
\end{equation}
\textbf{2ère colonne :}
calculer $H_2$ avec $\overrightarrow{x_2}$ puis :
\begin{equation}
    A_2=H_2\cdot A_1 = \begin{pmatrix}
        -\rho_1 & \alpha  & \beta  \\
        0       & -\rho_2 & \gamma \\
        0       & 0       & \delta
    \end{pmatrix}
    \nonumber
\end{equation}
$H_2$ devrait être de la forme :
\begin{equation}
    H_2=\begin{pmatrix}
        1 & 0     & 0     \\
        0 & \star & \star \\
        0 & \star & \star
    \end{pmatrix}
    \nonumber
\end{equation}
\textbf{Résoudre le système :}
\begin{equation}
    Q^T = H_2\cdot H_1
    \nonumber
\end{equation}
\begin{equation}
    R = A_2
    \nonumber
\end{equation}
\begin{equation}
    \begin{tabular}{ccc}
        $R\cdot \overrightarrow{x}=Q^T\cdot \overrightarrow{b}$
         & $\Longleftrightarrow $
         & $A_2\cdot \overrightarrow{x}=H_2\cdot H_1\cdot \overrightarrow{b}$
    \end{tabular}
    \nonumber
\end{equation}
\subsubsection*{Méthode de Jacobi}
\noindent
Pour les méthodes itératives, il faut que la matrice soit diagonalement dominante :
\begin{equation}
    \left | a_{ii} \right | > \sum_{j=1,j\neq i}^{n} \left | a_{ij} \right |
    \nonumber
\end{equation}
Soit le système :
\begin{equation}
    \begin{pmatrix}
        10 & 1  \\
        2  & 10 \\
    \end{pmatrix}
    \cdot
    \begin{pmatrix}
        x_1 \\
        x_2
    \end{pmatrix}
    =
    \begin{pmatrix}
        11 \\
        12 \\
    \end{pmatrix}
    \nonumber
\end{equation}
Qu'on réécrit sous la forme :
\begin{equation}
    \left\{\begin{aligned}
         & 10\cdot x_1^{(k+1)}=-1\cdot x_2^{(k)}+11 \\
         & 10\cdot x_2^{(k+1)}=-2\cdot x_1^{(k)}+12
    \end{aligned}\right.
    \nonumber
\end{equation}
\begin{equation}
    \left\{\begin{aligned}
         & x_1^{(k+1)}=\frac{1}{10}\cdot(-1\cdot x_2^{(k)}+11) \\
         & x_2^{(k+1)}=\frac{1}{10}\cdot(-2\cdot x_1^{(k)}+12)
    \end{aligned}\right.
    \nonumber
\end{equation}
Pour estimer le nombre d'itérations à effectuer afin de réduire l'erreur initiale
d'un facteur $\tau$ il faut calculer le rayon spectral de la matrice d'itération
$E=I-NA$, dans le cas de la méthode de Jacobi :
\begin{equation}
    \varrho(E)=\max_i\left |\lambda_i(E) \right |
    \nonumber
\end{equation}
\begin{equation}
    E=I-D^{-1}\cdot A
    \nonumber
\end{equation}
\begin{equation}
    n \geq \frac{\ln(\tau)}{\ln(\varrho(E))}
    \nonumber
\end{equation}
\subsubsection*{Méthode de Gauss-Seidel}
\noindent
Même chose que pour la méthode de Jacobi, mais on utilise les nouvelles valeurs
dans la même itération :
\begin{equation}
    \left\{\begin{aligned}
         & x_1^{(k+1)}=\frac{1}{10}\cdot(-1\cdot x_2^{(k)}+11)   \\
         & x_2^{(k+1)}=\frac{1}{10}\cdot(-2\cdot x_1^{(k+1)}+12)
    \end{aligned}\right.
    \nonumber
\end{equation}
Matrice d'itérations pour réduire l'erreur initiale d'un facteur $\tau$ :
\begin{equation}
    E = I-(D+L)^{-1}\cdot A
    \nonumber
\end{equation}
\subsubsection*{Méthode de relaxation}
\noindent
Même chose que pour la méthode de Gauss-Seidel, mais on rajoute un poids $\omega$ :
\begin{equation}
    \left\{\begin{aligned}
         & x_1^{(k+1)}=(1-\omega)\cdot x_1^{(k)}+\frac{\omega}{10}\cdot(-1\cdot x_2^{(k)}+11)   \\
         & x_2^{(k+1)}=(1-\omega)\cdot x_2^{(k)}+\frac{\omega}{10}\cdot(-2\cdot x_1^{(k+1)}+12)
    \end{aligned}\right.
    \nonumber
\end{equation}
\subsubsection*{Méthode de descente pour \boldmath{$F(\overrightarrow{x})=\left \| \overrightarrow{b}-A \overrightarrow{x} \right \|_2^2$}}
\noindent
On veut minimiser la fonction $F(\overrightarrow{x})$.\\
$\overrightarrow{x_0}$ étant le vecteur de départ (aléatoire), on calcule :\\
\textbf{L'erreur}
\begin{equation}
    F(\overrightarrow{x_0})=\left \| \overrightarrow{b}-A\cdot \overrightarrow{x_0} \right \|_2^2
    \nonumber
\end{equation}
\textbf{Le résidu}
\begin{equation}
    \overrightarrow{r_0}=\overrightarrow{b}-A\cdot \overrightarrow{x_0}
    \nonumber
\end{equation}
\textbf{Le gradient de \boldmath{$F(\overrightarrow{x})$}}
\begin{equation}
    \nabla F(\overrightarrow{x_0})=-2\cdot A^T\cdot \overrightarrow{r_0}
    \nonumber
\end{equation}
\textbf{La direction de descente}
\begin{equation}
    \overrightarrow{d_0}=-\nabla F(\overrightarrow{x_0})=2\cdot A^T\cdot \overrightarrow{r_0}
    \nonumber
\end{equation}
\textbf{Le pas}
\begin{equation}
    \alpha_0=\frac{\overrightarrow{r_0}^T\cdot A\cdot \overrightarrow{d_0}}{\overrightarrow{d_0}^T\cdot A^T\cdot A\cdot \overrightarrow{d_0}}
    \nonumber
\end{equation}
\textbf{Le nouveau vecteur}
\begin{equation}
    \overrightarrow{x_1}=\overrightarrow{x_0}+\alpha_0\cdot \overrightarrow{d_0}
    \nonumber
\end{equation}
Puis recommencer pour $\overrightarrow{x_1}$, $\overrightarrow{x_2}$, $\dots$\\
\subsubsection*{Méthode de descente pour \boldmath{$G(\overrightarrow{x})=\frac{1}{2} \overrightarrow{x}^T A \overrightarrow{x}-\overrightarrow{x}^T\overrightarrow{b}$}}
\noindent
On veut minimiser la fonction $G(\overrightarrow{x})$.\\
$\overrightarrow{x_0}$ étant le vecteur de départ (aléatoire), on calcule :\\
\textbf{L'erreur}
\begin{equation}
    G(\overrightarrow{x_0})=\frac{1}{2}\cdot \overrightarrow{x_0}^T\cdot A\cdot \overrightarrow{x_0}-\overrightarrow{x_0}^T\cdot \overrightarrow{b}
    \nonumber
\end{equation}
\textbf{Le gradient de \boldmath{$G(\overrightarrow{x})$}}
\begin{equation}
    \nabla G(\overrightarrow{x_0})=A\cdot \overrightarrow{x_0}-\overrightarrow{b}
    \nonumber
\end{equation}
\textbf{Le résidu et la direction de descente}
\begin{equation}
    \overrightarrow{r_0}=\overrightarrow{d_0}=\overrightarrow{b}-A\cdot \overrightarrow{x_0}
    \nonumber
\end{equation}
\textbf{Le pas}
\begin{equation}
    \alpha_0=\frac{\overrightarrow{d_0}^T\cdot \overrightarrow{d_0}}{\overrightarrow{d_0}^T\cdot A\cdot \overrightarrow{d_0}}
    \nonumber
\end{equation}
\textbf{Le nouveau vecteur}
\begin{equation}
    \overrightarrow{x_1}=\overrightarrow{x_0}+\alpha_0\cdot \overrightarrow{d_0}
    \nonumber
\end{equation}
Puis recommencer pour $\overrightarrow{x_1}$, $\overrightarrow{x_2}$, $\dots$\\
\subsubsection*{Méthode du gradient conjugués}
\noindent
On veut minimiser la fonction $G(\overrightarrow{x})$.\\
\textbf{L'erreur}
\begin{equation}
    G(\overrightarrow{x_k})=\frac{1}{2}\cdot \overrightarrow{x_k}^T\cdot A\cdot \overrightarrow{x_k}-\overrightarrow{x_k}^T\cdot \overrightarrow{b}
    \nonumber
\end{equation}
\textbf{Le gradient de \boldmath{$G(\overrightarrow{x})$}}
\begin{equation}
    \nabla G(\overrightarrow{x_k})=A\cdot \overrightarrow{x_k}-\overrightarrow{b}
    \nonumber
\end{equation}
\textbf{Le résidu}
\begin{equation}
    \overrightarrow{r_k}=\overrightarrow{b}-A\cdot \overrightarrow{x_k}
    \nonumber
\end{equation}
\textbf{La direction de descente}
\begin{equation}
    \left\{\begin{aligned}
         & d_0=r_0                                                                                                           \\
         & d_k = r_k+\beta_k\cdot d_{k-1}=r_k+\frac{\left \| r_k \right \|_2^2}{\left \| r_{k-1} \right \|_2^2}\cdot d_{k-1}
    \end{aligned}\right.
    \nonumber
\end{equation}
\textbf{Le pas}
\begin{equation}
    \alpha_0=\frac{\overrightarrow{d_k}^T\cdot \overrightarrow{r_k}}{\overrightarrow{d_k}^T\cdot A\cdot \overrightarrow{d_k}}
    \nonumber
\end{equation}
\textbf{Le nouveau vecteur}
\begin{equation}
    \overrightarrow{x}_{k+1}=\overrightarrow{x}_{k}+\alpha_k\cdot \overrightarrow{d}_k
    \nonumber
\end{equation}


